\documentclass[11pt]{article}
\usepackage[a4paper,margin=1in]{geometry}
\usepackage{amsmath,amssymb}
\usepackage{hyperref}
\usepackage{graphicx}
\usepackage{booktabs}
\usepackage{siunitx}

\title{Unified Simple Field Theory: Toy Particle--Universe Report}
\author{Generated Summary}
\date{\today}

\begin{document}
\maketitle

\section{Overview}
This report summarizes a self-contained toy ``particle universe'' model that encodes a compact closure for certainty-driven interactions, spin-induced torsion, two-sheet dynamics, and horizon drift. The model is not intended to be general relativity or Einstein--Cartan theory; instead it is a configurable sandbox for intuition building with explicit terms.

\section{State Variables and Dimensions}
Each particle carries:
\begin{itemize}
  \item A 6D spatial position $x_i \in \mathbb{R}^6$ and velocity $v_i \in \mathbb{R}^6$.
  \item A scalar uncertainty $\sigma_i$ (with floor/ceiling bounds).
  \item A scalar spin phase $s_i \in [-\pi, \pi)$.
  \item A sheet label (matter vs backsheet) controlling sign-flipped coupling.
\end{itemize}
Observables and diagnostics are presented in a 2D projection of the full state, which represents the effective 4D ``view'' of a single observer.

\section{Core Definitions}
Certainty and heat are defined as:
\begin{align}
C_i &= \frac{1}{\sigma_i^2},\\
H_i &= \frac{1}{C_i} = \sigma_i^2.
\end{align}
A buoyancy factor modulates coupling strength and sign:
\begin{equation}
 b_i = 1 - \frac{\sigma_i}{\sigma_{\text{crit}}}.
\end{equation}

\section{Forces}
The acceleration on particle $i$ is the sum of: (1) certainty-gradient gravity, (2) spin torsion, (3) optional black-hole frame dragging, and (4) mild velocity damping.

\subsection{Certainty-Gradient Gravity}
Interactions are driven by gradients in certainty rather than absolute certainty products. For each pair $(i,j)$:
\begin{equation}
 w_{ij} = G_c\, (C_i - C_j)\, b_i b_j\, s_{ij}\, g_{\text{bh}}(j),
\end{equation}
where $s_{ij}$ applies same-sheet attraction and cross-sheet sign flips, and $g_{\text{bh}}$ boosts attraction for low-$\sigma$ (``BH-like'') cores.

The resulting radial force is proportional to $w_{ij}/r_{ij}^3$.

\subsection{Spin Torsion}
Spin differences generate a transverse force in the projected $(x,y)$ plane:
\begin{equation}
 \Delta s_{ij} = s_i - s_j, \qquad
 \mathbf{a}^{\perp}_i \propto \sin(\Delta s_{ij})\, \frac{\mathbf{r}_{ij}^{\perp}}{r_{ij} + \epsilon},
\end{equation}
with a smooth core profile to limit short-range divergence.

\section{Two-Sheet Fabric and Time Reversal}
The system includes a ``backsheet'' (antimatter) that evolves with a flipped time direction. Cross-sheet interactions are sign-reversed and reduced, representing an opposite geometric response on the second sheet.

\section{Horizon Drift}
A terminal drift pushes particles outward toward a boundary at radius $R$:
\begin{equation}
 \mathbf{v}_{\text{drift}} = v_0\, \frac{y}{(1-y)^{\alpha}}\, \hat{r}, \qquad y = \frac{r}{R}.
\end{equation}
This produces a dark-energy-like outward flow near the horizon.

\section{Thermodynamics}
Uncertainty evolves via cooling toward a floor and heating from local kinetic energy:
\begin{equation}
 \dot{\sigma}_i = -\lambda(\sigma_i-\sigma_{\text{floor}}) + \eta\, \lVert v_i \rVert^2,
\end{equation}
with clamping to $[\sigma_{\text{floor}},\sigma_{\text{ceiling}}]$.

\section{Diagnostics}
Optional diagnostics report:
\begin{itemize}
  \item Mean kinetic energy per particle.
  \item Mean certainty and mean heat.
  \item RMS radius.
\end{itemize}
These allow rough comparison to baseline or standard models (e.g., Newtonian N-body) by tracking global trends over time.

\section{Key Parameters}
The model centralizes configuration in a parameter bundle, including:
\begin{itemize}
  \item $G_c$, $\sigma_{\text{crit}}$, $\epsilon$ (certainty gravity core settings).
  \item Spin torsion strength and scale.
  \item Horizon radius and drift parameters.
  \item Thermodynamic rates and $\sigma$ distribution parameters ($\mu$, $\sigma_{\text{std}}$).
\end{itemize}

\section{Notes and Intended Use}
This is a compact closure with explicit terms, designed to keep complexity minimal and to emphasize the qualitative roles of certainty gradients, spin torsion, and horizon drift. It is intended as a sandbox for exploration rather than a statement of physical theory.

\end{document}
